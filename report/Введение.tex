\section*{ВВЕДЕНИЕ}
\addcontentsline{toc}{section}{ВВЕДЕНИЕ}

В современном мире данные являются одним из самых ценных ресурсов, что делает их обработку и управление ключевыми задачами для различных сфер деятельности. СУБД обеспечивают эффективное хранение, обработку, управление и доступ к информации, играя важную роль в бизнесе, науке, образовании и других областях.

Разработка СУБД представляет собой сложный процесс, включающий анализ предметной области, проектирование архитектуры системы и её реализацию с использованием современных технологий программирования. Такая система позволяет автоматизировать многие рутинные процессы, улучшить производительность и сократить вероятность ошибок при работе с большими объемами данных.

Главной задачей профессионально построенной системы управления базами данных является корректное хранение данных и быстрый доступ к ним.

\emph{Цель настоящей работы} – создание функциональной системы управления базами данных, ориентированной на практическое применение в реальных условиях. СУБД, которая обеспечивает поддержку основных операций с данными, включая их добавление, удаление, обновление и поиск. Для достижения поставленной цели необходимо решить \emph{следующие задачи:}
\begin{itemize}
\item провести анализ предметной области;
\item разработать концептуальную модель СУБД;
\item реализовать функционал СУБД средствами современных технологий программирования;
\item провести тестирование и анализ работы системы.
\end{itemize}

\emph{Структура и объем работы.} Отчет состоит из введения, 4 разделов основной части, заключения, списка использованных источников, 2 приложений. Текст выпускной квалификационной работы равен \formbytotal{lastpage}{страниц}{е}{ам}{ам}.

\emph{Во введении} сформулирована цель работы, поставлены задачи разработки, описана структура работы, приведено краткое содержание каждого из разделов.

\emph{В первом разделе} на стадии описания технической характеристики предметной области приводится сбор информации о устройстве системы управления базами данных.

\emph{Во втором разделе} на стадии технического задания приводятся требования к разрабатываемой СУБД.

\emph{В третьем разделе} на стадии технического проектирования представлены проектные решения для СУБД.

\emph{В четвертом разделе} приводится список макроссов и функций, использованных при разработке СУБД, производится тестирование разработанной системы.

В заключении излагаются основные результаты работы, полученные в ходе разработки.

В приложении А представлены фрагменты исходного кода. 
В приложении Б представлены графические компоненты.
