\section{Рабочий проект}
\subsection{Структуры, используемые при разработке СУБД}

Можно выделить следующий список структур и их методов, использованных при разработке СУБД (таблица \ref{class:table}).

\renewcommand{\arraystretch}{0.8} % уменьшение расстояний до сетки таблицы
\begin{xltabular}{\textwidth}{|X|p{2.5cm}|>{\setlength{\baselineskip}{0.7\baselineskip}}p{4.85cm}|>{\setlength{\baselineskip}{0.7\baselineskip}}p{4.85cm}|}
\caption{Описание структур, используемых в СУБД\label{class:table}}\\
\hline \centrow \setlength{\baselineskip}{0.7\baselineskip} Название структуры & \centrow \setlength{\baselineskip}{0.7\baselineskip} Модуль, к которому относится структура & \centrow Описание структуры & \centrow Методы \\
\hline \centrow 1 & \centrow 2 & \centrow 3 & \centrow 4\\ \hline
\endfirsthead
\caption*{Продолжение таблицы \ref{class:table}}\\
\hline \centrow 1 & \centrow 2 & \centrow 3 & \centrow 4\\ \hline
\finishhead
table & main & table – основная структура приложения. Имеет такие поля как: name, rows, columns. & Нет.\\
\end{xltabular}
\renewcommand{\arraystretch}{1.0} % восстановление сетки

\subsection{Описание функций системы управления базами данных}

\begin{enumerate}
\item create-table -- функция для создания таблицы. 
\item insert-into -- функция для вставки данных в таблицу.
\item make-select-from -- макрос для выборки данных по различным критериям.
\item delete-from -- функция для удаления данных из таблицы.
\item drop-table -- функция для удаления таблицы.
\item backup-table -- функция для создании копии таблицы в другую таблицу.
\item save-table-to-file -- функция для записи таблицы в другой файл.
\item make-update -- макрос для обновления данных по различным критериям.
\end{enumerate}

\subsection{Автоматизированное тестирование СУБД}

Один из автоматизированных тестов функций СУБД представлен на рисунке \ref{unittest:image}.

\begin{figure}[ht]
\begin{lstlisting}[language=Lisp]
(defun test-select-from (table-name columns rows selected-columns expected-result &optional condition)
(create-table table-name columns)
(dolist (row rows)
(insert-into table-name row))
(let ((result (select-from table-name selected-columns condition)))
(as-eq "test-select-from" result expected-result)))
\end{lstlisting}  
\caption{Автоматизированный тест функции}
\label{unittest:image}
\end{figure}

Вызов теста может осуществляться с разными входными данными. Пример представлен на рисунке \ref{unittestEx:image}.

\begin{figure}[ht]
\begin{lstlisting}[language=Lisp]
		(test-select-from 'test-table1211
		'((id integer) (name string) (age integer))
		'(((id . 1) (name . "egor") (age . 19))
		((id . 2) (name . "admin") (age . 24))
		((id . 3) (name . "alex") (age . 26)))
		'(name)
		'(((name . "admin")))
		'((> age 23) (/= name "alex")))
\end{lstlisting}  
\caption{Вызов автоматизированного теста функции}
\label{unittestEx:image}
\end{figure}
