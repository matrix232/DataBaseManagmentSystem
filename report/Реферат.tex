\abstract{РЕФЕРАТ}

Объем работы равен \formbytotal{lastpage}{страниц}{е}{ам}{ам}. Работа содержит \formbytotal{figurecnt}{иллюстраци}{ю}{и}{й}, 1 таблицу, \arabic{bibcount} библиографических источников и \formbytotal{числоПлакатов}{лист}{}{а}{ов} исходного кода. Количество приложений – 2. Фрагменты исходного кода представлены в приложении А. Графические компоненты представлены в приложении Б.

Перечень ключевых слов: база данных, пользователь, система, система управления базами данных, данные, структура, макросы, функции, методы.

Объектом разработки является системы управления базами данных.
Целью курсовой работы является улучшения понимания работы системы управления базами данных и создание своей собственной базы данных.

При создании программного продукта применялась технология функционального программирования.

\selectlanguage{english}
\abstract{ABSTRACT}
  
The volume of work is \formbytotal{lastpage}{page}{}{s}{s}. The work contains \formbytotal{figurecnt}{illustration}{}{s}{s}, 1 table, \arabic{bibcount} bibliographic sources and \formbytotal{числоПлакатов}{sheet}{}{s}{s} of listing. Number of appendices – 2. Source code fragments in Appendix A. Source graphic components in Appendix B.

List of keywords: database, user, system, database management system, data, structure, macros, functions, methods.

The object of development is a database management system.
The purpose of the course work is to improve the understanding of the database management system and to create your own database.

The technology of functional programming was used in the creation of the software product.
\selectlanguage{russian}
