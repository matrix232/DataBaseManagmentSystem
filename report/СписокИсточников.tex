\addcontentsline{toc}{section}{СПИСОК ИСПОЛЬЗОВАННЫХ ИСТОЧНИКОВ}

\begin{thebibliography}{9}

    \bibitem{lisp1} Сайбель Питер, С.П. Практическое использование Common Lisp / С.П. Сайбель Питер. – М. СПб. : Питер, 2017. – 615 с. – ISBN 978-5-97060-538-7. – Текст~: непосредственный.
    \bibitem{lisp2} Грэм Пол, Г. П. ANSI Common Lisp / Г. П. Грэм Пол. – Москва : Бомбора, 2017. – 498 с. – ISBN 978-5-93286-206-3.. – Текст~: непосредственный.
    \bibitem{sql1} Шилдс, Ш. У. SQL: быстрое погружение / Ш. У. Шилдс. – М. СПб. : Питер, 2022. – 254 с. – ISBN 978-5-4461-1835-9.. – Текст~: непосредственный.
    \bibitem{sql2}	Моргунов, Е. П. PostgreSQL. Основы языка SQL / Е. П. Моргунов. – Москва : БХВ, 2022. – 336 с. – ISBN 978-5-9775-4022-3. – Текст~: непосредственный.
	\bibitem{arch}	Роберт, М. Чистая архитектура. Искусство разработки программного обеспечения / М. Роберт. – М. СПб. : Питер, 2022. – 592 с. – ISBN 978-5-4461-0772-8. – Текст~: непосредственный.
	\bibitem{arch2}	Head First. Паттерны проектирования. 2-е издание / Эрик Фримен, Элизабет Робсон, Кэтти Сьерра, Берт Бейтс. – М. СПб. : Питер, 2021. – 640 с. – ISBN 978-5-4461-1819-9. – Текст~: непосредственный.
	\bibitem{arch3}	Бхаргава, Адитья Грокаем алгоритмы. Иллюстрированное пособие для программистов и любопытствующих / Адитья Бхаргава. – М. СПб. : Питер, 2022. – 288 с. – ISBN 9785446109234. – Текст~: непосредственный.
	\bibitem{sql4}	Бондарь, А.Г. Microsoft SQL Server 2022 / А.Г. Бондарь. – Москва : БХВ, 2022. – 528 с. – ISBN 978-5-9775-1805-5. – Текст~: непосредственный.
	\bibitem{func}	Фридман, Д.П. THE LITTLE SCHEMER: Чудесное функциональное программирование / Д.П. Фридман. – Москва : Наука и техника, 2024. – 230 с. – ISBN 978-5-93700-234-1. – Текст~: непосредственный.    
	\bibitem{site}	Metanit : сайт. – URL: https://metanit.com/sql/mysql/2.1.php (дата обращения: 11.01.2025)    
	\bibitem{site2}	GitHub : сайт. – URL: https://github.com/sqlalchemy/sqlalchemy (дата обращения: 11.01.2025)    
\end{thebibliography}
