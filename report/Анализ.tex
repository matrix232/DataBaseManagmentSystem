\section{Анализ предметной области}
Характеристика и использование СУБД.

Системы управления базами данных (СУБД) начали формироваться с середины XX века. Первая коммерческая СУБД, известная как Integrated Data Store (IDS), была разработана в 1960-х годах инженером Чарльзом Бахманом. Она представляла собой навигационную СУБД, где данные организовывались в виде сетевых структур. 

В 1970 году революционным шагом стало предложение Эдгара Кодда, который разработал реляционную модель данных, опубликованную в его работе "A Relational Model of Data for Large Shared Data Banks". На основе этой модели в 1980-х годах появились первые реляционные СУБД, такие как IBM System R, Oracle и другие, которые до сих пор составляют основу многих современных систем.

 Предметная область включает в себя процессы хранения, обработки, извлечения и анализа данных, которые актуальны для различных организаций и сфер деятельности. СУБД используется для управления информацией в самых разных контекстах, таких как коммерческие компании, научные учреждения, государственные структуры и образовательные организации. В основе всех этих процессов лежит необходимость упорядоченного хранения данных, которые могут быть представлены в виде таблиц, графов или других структур. Поэтому ключевыми аспектами анализа предметной области являются: 
 
 \begin{enumerate}
 \item Определение типов данных. Для разработки СУБД важно учитывать, какие данные будут храниться в системе.
 \item Выявление пользователей и их ролей. Анализ должен определить, кто будет работать с СУБД, какие операции они смогут выполнять, и какие уровни доступа к данным им необходимы.
 \item Требования к безопасности данных. Важно предусмотреть защиту информации от несанкционированного доступа, утечек и потерь.
 \item Интеграция с другими системами. Если СУБД должна взаимодействовать с другими программными продуктами, необходимо определить требования к интерфейсам и протоколам обмена данными.
 \end{enumerate}

